\title{оказаться, оказаться и попасть}
\begin{document}
\section{}
Сегодня я услышал от ребёнка "оказаться в аварии".
 По-русски так сказать неправильно, в аварию можно только \em{попасть}.
Однако же у \em{оказаться} и \em{попасть} — похожая сочетаемость: для многих имен \em{X}, с которыми возможна конструкция \em{оказаться в X}, возможна и \em{попасть в X}, в то же время сочетаемость у этих глаголов не произвольная — в большинство имен русского языка \em{попасть} нельзя, нельзя в них и \em{оказаться}.

Если этим можно объяснить ошибку ребёнка, то точнее можно сформулировать так: есть корреляция между употребимостью слова \em{X} в конструкции \em{оказаться в X} с употребимостью в \em{попасть в X}.

Теперь можно посмотреть в корпусе конкретные числа и проверить.
\section{}
Так почему же это неправильно?
Гипотеза: в русском языке есть несколько разных \em{оказаться}, каждый со своей сочетаемостью и моделью управления. Они, должно быть, как то различаются по значению, иначе что вообще значит "существуют несколько разных оказаться"?.

\end{document}